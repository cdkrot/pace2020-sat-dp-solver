\documentclass[a4paper,UKenglish,cleveref, autoref, thm-restate]{lipics-v2019}
%This is a template for producing LIPIcs articles. 
%See lipics-manual.pdf for further information.
%for A4 paper format use option "a4paper", for US-letter use option "letterpaper"
%for british hyphenation rules use option "UKenglish", for american hyphenation rules use option "USenglish"
%for section-numbered lemmas etc., use "numberwithinsect"
%for enabling cleveref support, use "cleveref"
%for enabling autoref support, use "autoref"
%for anonymousing the authors (e.g. for double-blind review), add "anonymous"
%for enabling thm-restate support, use "thm-restate"

%\graphicspath{{./graphics/}}%helpful if your graphic files are in another directory

\bibliographystyle{plainurl}% the mandatory bibstyle

\title{Sat and Dynamic Programming based Tree Depth Solver}

\author{Dmitry Sayutin}{ITMO University, Russia, St. Petersburg \and Higher School of Economics, St. Petersburg campus, Russia \and \url{cdkrot.me}}
{%email
}{%orcid
}{}

%\author{John Q. Public}{Dummy University Computing Laboratory, [optional: Address], Country \and My second affiliation, Country \and \url{http://www.myhomepage.edu} }{johnqpublic@dummyuni.org}{https://orcid.org/0000-0002-1825-0097}{(Optional) author-specific funding acknowledgements}%TODO mandatory, please use full name; only 1 author per \author macro; first two parameters are mandatory, other parameters can be empty. Please provide at least the name of the affiliation and the country. The full address is optional

\authorrunning{D. Sayutin}

\Copyright{Dmitry Sayutin}

\ccsdesc[100]{Theory of computation~Design and analysis of algorithms~Graph algorithms analysis}

\keywords{tree depth, sat, dynamic programming}  %TODO mandatory; please add comma-separated list of keywords

\supplement{https:/doi.org/10.5281/zenodo.3888908}
\supplement{https://github.com/cdkrot/pace2020-sat-dp-solver} %optional, e.g. related research data, source code, ... hosted on a repository like zenodo, figshare, GitHub, ...

% \acknowledgements{I want to thank \dots}%optional

\nolinenumbers %uncomment to disable line numbering

%\hideLIPIcs  %uncomment to remove references to LIPIcs series (logo, DOI, ...), e.g. when preparing a pre-final version to be uploaded to arXiv or another public repository

%Editor-only macros:: begin (do not touch as author)%%%%%%%%%%%%%%%%%%%%%%%%%%%%%%%%%%
\EventEditors{John Q. Open and Joan R. Access}
\EventNoEds{2}
\EventLongTitle{42nd Conference on Very Important Topics (CVIT 2016)}
\EventShortTitle{CVIT 2016}
\EventAcronym{CVIT}
\EventYear{2016}
\EventDate{December 24--27, 2016}
\EventLocation{Little Whinging, United Kingdom}
\EventLogo{}
\SeriesVolume{42}
\ArticleNo{23}
%%%%%%%%%%%%%%%%%%%%%%%%%%%%%%%%%%%%%%%%%%%%%%%%%%%%%%

\begin{document}

\maketitle

\begin{abstract}
This article describes sat-based and dp-based approaches as used in Magnolia solver submitted for PACE 2020
\end{abstract}

\section{Introduction}

\begin{definition}
  Given connected undirected graph $G = (V, E)$, we say that rooted tree $T$ with the same vertex set $V$ is a tree-depth decomposition
  if for any edge $\{x, y\} \in E$ the corresponding vertices are bound by an ancestry relation in $T$ (that is, either $x$ is ancestor
  of $y$ in $T$ or vice versa).
\end{definition}

\begin{definition}
  The tree depth of the graph $G$ is the smallest possible height of the $G$'s tree decomposition.
\end{definition}

It is clear that this definition is well-defined, since the chain graph of vertices in $V$ (rooted at one of its ends) serves as a valid
tree-depth decomposition. Tree-depth decomposition will be called \textit{optimal} if it has the smallest possible height. There might be multiple
optimal tree-depth decomposition for a graph.

The remaining part of this text focuses on discussion of algorithms computing tree depth (and finding optimal tree depth decomposition as well).

\section{Dynamic Programming -- simple, but not useless}

One of the most simple ways to solve the problem is the following dynamic programming on the subsets.

Let $\texttt{dp}(S) \colon 2^V \to \mathbb{Z}$ be the tree-depth of $G[S]$, where $G[S]$ denotes
induced subgraph to the subset of vertices of $V$.

It can be computed as follows:

$$\texttt{dp}(S) = \begin{cases}
  0 & if~S = \varnothing \\
  \min_{v \in S} (1 + \max\limits_{c \in \texttt{cc}(G[S] \setminus \{v\})} \texttt{dp}(c)) & otherwise \\
\end{cases}
$$

Here $\texttt{cc}(G)$ denotes set of vertex sets of connected components of $G$. Careful reader will
probably note, that $G[S]$ might not be connected, so its unclear what $\texttt{dp}(S)$ for such $S$
should mean. However, if we calculate dp values lazily -- that is, if we calculate only the states
reachable from $\texttt{dp}(V)$, we will always stay in a single connected component. So for the
further discussion we simply assume that all mentioned $G[S]$ are non-empty and connected.

It's easy to make an induction proof that this dynamic programming works correctly.
Also given values of all intermediate states we can compute the tree depth decomposition itself.
The complexity of this solution is $\mathcal{O}(2^n poly(n))$ where polynomial factor depends on the implementation.
This is rather slow, however it's possible to prune transitions in dp, to make it work faster. Consider the following
two lemmas.

\begin{lemma} In case $G[S]$ contains a universal vertex, that is vertex adjacent to every other vertex in $G[S]$, we
  can only consider transition by this vertex for the set $S$.

  In other words, there exists an optimal tree depth decomposition having this vertex as root.
\end{lemma}

\begin{proof} We can argue by contradiction: suppose there are no such tree depth decompositions. Examine
  arbitrary optimal tree depth decomposition $T$ for $G[S]$.

  Let's denote universal vertex as $r$. Observe that all other vertices of $G[S]$ must either be ancestors of $r$ or
  descendands of $r$. Then if we consider unique path from parent of $r$ to the tree root in $T$, there can't be any branching
  vertices on it. So if we swap $r$ and root of $T$ (but preserve the tree structure) we can optain another optimal tree depth decomposition.
\end{proof}

\begin{lemma} In case the $G[S]$ \textbf{doesn't} contain a universal vertex (we can't apply previous lemma), we can skip
  transitions for all leaf vertices in $G[S]$, that is for all vertices with $\texttt{deg} = 1$ in the induced subgraph.

  In other words, there exists an optimal tree depth decomposition such that its root is not a leaf.
\end{lemma}

\begin{proof} Suppose the optimal tree depth decomposition $T$ is rooted at leaf vertex $v$, and the only vertex adjacent to $v$ is $u$.

  Now we will show, that there is another optimal tree depth decomposition rooted at $u$. Note that this implies the lemma,
  since $u$ can't be leaf vertex (unless $G[S]$ consists of two vertices only, however in this case there are two universal vertices).

  Observe that due to $G[S] - v$ being connected, there can be only one neighbor of $v$ in $T$ (though it can be other vertex than $u$).

  Construct the tree $T'$ by swapping vertices $v$ and $u$, while preserving the tree structure. Vertices $v$ and $u$ are still in ancestry
  relation, and any two vertices in $G[S] - v$ having an ancestry relation in $T$  are in an ancestry relaion in $T'$ as well. So $T'$ is another
  optimal tree-depth decomposition of $G[S]$, with root $u$.
\end{proof}

Those lemmas naturally lead to the faster solution by drastically reducing the number of visited states. Unfortunately this
approach doesn't improve the worst-case asymptotics for the number of visited states.

This solution can be implemented particularly efficiently in case we assumpe computation model of RAM-machine
and that $n \le w$ (where $w$ denotes the number of bits in machine word).

Those two lemmas are inspired by similar facts proved in \cite{sanchez2017treedepth}.

\section{SAT-based solution}

This solution implements ideas from \cite{ganian2019sat}.

Let's fix an upper bound $L$ on the tree depth. Then we need to decide whether there exists a tree decomposition of
height at most $L$.

We will solve this decision problem by applying combinatorial reduction to SAT and then simply calling SAT-solver.

Given a tree-depth decomposition $T$ with height at most $L$, we can construct a sequence $P_0, P_1, \ldots, P_L$
of partitions of the graph vertex set. Where $P_i$ is as follows: suppose we delete all vertices in a
tree-depth decomposition with depth at most $L - i$ (here, depth of the root is $1$) and partition vertices based on
connected components. Observe that $P_L$ should have a single connected component, while for $P_0$ all vertices
are deleted. 

In order to work with such partitions we can introduce an indicator variables $f(v, u, i)$ denoting
whether vertices $v$ and $u$ are not deleted and are in the same component in $P_i$.
We will assume, that $f(v, u, i)$ and $f(u, v, i)$ refer to the same variable.

The variable assignment can easily be derived from the tree-depth decomposition. Now we will add some constraints,
so SAT solution corresponds to almost tree structure as well.

In the list below, unless said otherwise,
assume $v, u, w$ range over all possible vertices, and $i$ ranges from $0$ to $x$ inclusively,

\begin{alphaenumerate}
\item If we fix $i$, we should have an equivalency relation (on non-deleted vertices):
  
  $f(v, w, i) \lor \lnot f(v, u, i) \lor \lnot f(u, w, i)$ (that is, $f(v, u, i) \land f(u, w, i) \implies f(v, w, i)$.

\item If $i = L$ there is a single connected component, and when $i=0$ there should be no vertices.

  $f(v, u, L) = 1$, $f(v, u, 0) = 0$.

\item The connected components are growing when iterating $i \to i+1$:

  $f(v, u, i) \lor \lnot f(v, u, i - 1)$, where $1 \le i \le L$.

  In other words, $f(v, u, i - 1) \implies f(v, u, i)$.

\item In case two vertices appear in a current layer, they are not connected yet.

  $f(v, v, i - 1) \lor f(u, u, i - 1) \lor \lnot f(v, u, i)$, where $1 \le i \le L$.

  In other words, $\lnot f(v, v, i - 1) \and \lnot f(u, u, i - 1) \implies \lnot f(v, u, i)$.
\end{alphaenumerate}

Conditions above ensure we've built a well-formed tree. Formally, the conditions above hold for any possible
partition sequence derived from decomposition tree, and for any variable assignment satisfying (a)--(d), we
can easily build a tree from it. There is a small caveat that we may get partition sequence in which some
vertices are introduced at larger depth than one naturaly would do, but this is acceptable for our goals.

Now we need to ensure that endpoints of every edge $\{a, b\} \in E$ are in an ancestry relation.
This can be done by the following conditions.
\begin{enumerate}[(e)]
\item $\lnot f(a, a, i) \lor \lnot f(b, b, i) \lor f(a, a, i - 1) \lor f(a, b, i)$
	
	and vice versa: $\lnot f(a, a, i) \lor \lnot f(b, b, i) \lor f(b, b, i - 1) \lor f(a, b, i)$

    For $\{a, b\} \in E$, $1 \le i \le L$.
\end{enumerate}

% apparently, weaker condition is sufficient, though moron myself hasn't submitted it
% \lnot f(a, a, i) \lor \lnot f(b, b, i) \lor f(a, b, i)$

\subsection{SAT solution, practical matters}

Given conditions above, one can choose good state-of-the-art SAT solver.

And since the SAT reduction only decides whether the answer is at most $L$, there is a question on how it is better
to reduce search of minimum tree-depth to it. Out of three methods: linear search, binary search and adaptive binary search,
the latter works the best, though for some instances linear search is better.

\begin{lstlisting}[caption={Linear Search},label=list:8-6,captionpos=t,abovecaptionskip=-\medskipamount]
ans = 1
while (not decide(ans))
  ++ans
print(ans)
\end{lstlisting}  

\begin{lstlisting}[caption={Binary Search},label=list:8-6,captionpos=t,abovecaptionskip=-\medskipamount]
L = 0, R = n // L is impossible, R is possible
while (R - L > 1) {
  M = (L + R) / 2
  if (decide(M))
    L = M
  else
    R = M
print(R)
\end{lstlisting}

\begin{lstlisting}[caption={Adaptive Binary Search},label=list:8-6,captionpos=t,abovecaptionskip=-\medskipamount]
L = 0, R = 1
while (not decide(R))
  L = R, R = 2 * R

// the rest is as in binary search
while (R - L > 1) {
  M = (L + R) / 2
  if (decide(M))
    L = M
  else
    R = M
print(R)
\end{lstlisting}  

In adaptive binary search both the number of SAT-solver invocations is reduced and the instance size given to SAT is kept small.

%%
%% Bibliography
%%

\bibliography{article}

\end{document}
